%% -*- mode: latex; mode: visual-line; mode: flyspell -*-
\documentclass{article}


\usepackage{array}
\usepackage{graphicx}
\usepackage{amsmath}
\usepackage{times}
\renewcommand\ttdefault{aett}

\newcommand\pterm[1]{\multicolumn{2}{l}{#1}\\}
\newcommand\pfeqonly[1]{%
   \mskip 15mu{}={} & \{\mbox{#1}\} \\
}
\newcommand\pfeq[2][]{\pfeqonly{#1}\pterm{#2}}
\newcommand\ptermmbox[1]{\multicolumn{2}{l}{\mbox{#1}}\\}
\newenvironment{codeproof}{\[\advance\extrarowheight by 2pt 
                             \let\pterm=\ptermmbox\begin{array}{c@{}l}}
                     {\end{array}\]\ifhmode\unskip\par\fi\csname @endparenv\endcsname}

\let\Tt=\ttfamily
\let\nwendquote=\relax


\begin{document}


\parskip=0.8\baselineskip plus 2pt
\parindent=0pt

Here are some examples of LaTeX commands that could help you write a
calculational proof.

Writing a code chunk on a line: \texttt{(length (append xs ys))}.

And here's a chunk of a calculational proof:
\begin{codeproof}
\pterm{{\Tt{}(length (append xs ys))\nwendquote}}
\pfeqonly{ by assumption, $\mathtt{xs} = \mbox{\texttt{(cons z zs)}}$ }
\pterm{\Tt{}(length (append (cons z zs) ys))}
\pfeqonly{ append-cons law }
\pterm{\Tt{}(length (cons z (append zs ys))}
\end{codeproof}


\end{document}
